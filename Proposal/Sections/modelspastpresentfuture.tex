Mathematical modelling has been a staple of sciences for centuries~\cite{banerjeeMathematicalModelingModels2021}. From using simple geometry to construct temples and calculate the distance of the Sun to the Earth, to simulate neutron transport in nuclear fission~\cite{demaziereModellingNuclearReactor2020}. This approach has also not been overlooked in the biological sciences, where models have been used to understand population dynamics~\cite{wangerskyLotkaVolterraPopulationModels1978}, enzyme kinetics~\cite{cornish-bowdenOneHundredYears2015}, and many more. Over time, these mathematical models have become more and more complex, which entail a difficulty in understanding them. Fortunately, humanity has developed more and more ways to communicate with complex mathematics by using actual computational power. This advancement has allowed modern sciences to create computational models that help with scientific advances.

There are many different types of computational models used in biological studies. In photosynthesis alone, some may be used to understand a specific part of the mechanism~\cite{matuszynskaMathematicalModelNonphotochemical2016,liImpactIonFluxes2021}. In contrast, others may be used to predict the behaviour and "output" of a plant~\cite{voncaemmererSteadystateModelsPhotosynthesis2013, zhuEphotosynthesisComprehensiveDynamic2013}. Some are built using kinetic knowledge, while others are based on stochastic processes. They all have key differences that make each unique and suitable for different purposes. While variety is the spice of life, this also introduces a significant complexity when trying to choose, understand, and use these models.

This "problem" has already been in the focus of the scientific community, therefore some proposed solutions already exist. These solutions tackle the problem from different angles, which can be separated in three different categories: Creation (Past), Presentation (Present), and Sharing (Future).

\subsection{Creation (Past)}

There are many different ways to create a computational model in biological studies. Some may be built completely from scratch, using programming languages like Python, or MATLAB. Others may be built using more specialised software developed for this kind of tasks, like MxLpy~\cite{vanaalstMxlPyPythonPackage2025}, COPASI~\cite{hoopsCOPASICOmplexPAthway2006}, or Tellurium~\cite{choiTelluriumExtensiblePythonbased2018}. Whatever the method, all have their own respective costs and benefits, however, one thing is common to all: the annotation and documentation of the model is based on the author of the model.

This annotation and documentation is key to understanding the model, as it allows the author and others to understand how, why, and what the model represents. However, as there are many different ways to create a model, there are also many ways to document it. There have been attempts to standardise this process, one of the most notable being the \gls{sbml}~\cite{huckaSystemsBiologyMarkup2003}. \gls{sbml} is an XML-based format for representing computational models in systems biology. It allows for the sharing and exchange of models between different software tools, making it easier for researchers to collaborate and build upon each other's work. Since its first publication in 2003~\cite{huckaSystemsBiologyMarkup2003}, the \gls{sbml} language has been updated and extended several times~\cite{SBMLorgEssentialPublications}. It has become a common way to represent biological computational models, with many more also available.

While the idea and reception of such standardised languages have been positive, they still carry some issues. Their implementation is not always straightforward and with every new version, new features are added that may not be backward compatible.

\subsection{Presentation (Present)}

Reusability and reproducibility of work has always been a core principle of scientific work, however even the sharing of data has fallen victim to poor practices. An analysis conducted by Roche et al.~\cite{rochePublicDataArchiving2015} by examining a dataset of 100 nonmolecular evolutionary or ecological articles between 2012 and 2013 that have been published in seven different leading journals found that 56\% of the articles archived incomplete data and furthermore 64\% of the articles were not fully reproducible. This has sparked the creation of the \gls{fair} Guiding principles~\cite{wilkinsonFAIRGuidingPrinciples2016}, which has since been extended to the \gls{cure} Guidelines~\cite{sauroFAIRCUREGuidelines2025}.

These guidelines propose a set of principles to ensure good scientific work that also extends to computational models. They encapsulate a varying range of aspects, ranging from how the model is created and validated, to how it is provided and documented. The \gls{fair} guidelines have already been widely adopted \figref{fig:fairUsage}, which is shown by the 7841 citations~\cite{CitationsCommentFAIR} (as of October 2025). Solely in 2024 the comment was cited 1408 times (as of October 2025), in which 69.2\% are articles. As the \gls{cure} guidelines are more recent, they have not yet seen the same adoption rate, but they are gaining traction in the community.

\begin{figure*}[!htb]
    \centering
    \includegraphics[width=0.7\textwidth]{Figures/fair_usage.pdf}
    \captionprof{The usage of the \glsxtrshort{fair} Guidelines found by citations.}{The citations of the original \glsxtrfull{fair} guiding principles comment~\cite{wilkinsonFAIRGuidingPrinciples2016}, from the years 2016 to 2026, were found. Additionally, the citations of the year 2024 have been separated into the different types of publications. Articles, Review, and proceeding papers have their own category, while others (editorial material, early access, data paper, etc.) are grouped together. Data taken from Web of Science on 29th October 2025.~\cite{CitationsCommentFAIR}}
    \label{fig:fairUsage}
\end{figure*}

\subsection{Sharing (Future)}

While computational models in biological studies rise of popularity is at surface level a very positive change, it does bring its own troubles with. The more models available to the public, the larger the pool of choices. As every model is released by using a publication, someone that wishes to use a model will have to sift through numerous publications to find a model that suits their needs. Solely in 2024, 1475 publications were published when querying for "photosynthesis model"~\cite{CitationReport1475} and the trend seems to continue upward every year~\figref{fig:photosynthesisModelQuery}. Due to this trend and accessibility issue, an idea for databases arose. The most notable one, called BioModels~\cite{malik-sheriffBioModels15Years2019}, already includes 1096 manually curated models and 1661 non-curated ones, totaling to 2757 different models.

\begin{figure*}[!htb]
    \centering
    \includegraphics[width=0.7\textwidth]{Figures/photosynthesis_model_query.pdf}
    \captionprof{Number of publications found by using "photosynthesis model" as a query.}{The number of publications found by querying "photosynthesis model" from the years 2000 to 2024, totaling to 20265 publications. Data taken from Web of Science on 11th November 2025.~\cite{CitationReport1475}}
    \label{fig:photosynthesisModelQuery}
\end{figure*}

BioModels does not restrict what type of model can be submitted to the database. Therefore, the assortment of models is very diverse and ranges between species, from \textit{Homo Sapiens} to \textit{Saccharomyces cerevisiae}, modelling approaches, from \gls{ode} to constraint-based, and modelling formats, from \gls{sbml} to Python. What is restricted, however are some aspects of model publication. BioModels has constructed their guidelines called \gls{miriam}, which focuses on consistent annotation and curation of computational models~\cite{novereMinimumInformationRequested2005}. To further their goal of easier access to models, the team behind BioModels also manually curate models to a general standard, with a prerequisite that these submitted models are written in \gls{sbml}. This effort has proven to be popular in the scientific community to many extents, as it not only makes sharing easier, but also encourages modellers to publish their model into the public domain, as BioModels does not take ownership of any submitted model. Even though this resource is hugely valuable to science, it still has some flaws.

A study done by Tiwari et al.~\cite{tiwariReproducibilitySystemsBiology2021} has analysed the reproducibility of 455 models taken from the BioModels database, which equates to over 20\% of literature-based models available on BioModels at the time of writing. They found out, that 49\% of the published models are not directly reproducible, even with the \gls{miriam}-following prerequisite of uploading to Biomodels. From these directly irreproducible models, 25\% (54 models, which is 12\% from the original 455 models) could be saved by manual empirical correction or author support. At the end, about 37\% of the total analysed models (equates to 169 models) from the BioModels database could not be reproduced at all. The common issues that sometimes could be manually repaired, were the missing of critical model information, wrongly described equations, or other small, easily overseen, mistakes, that can completely destroy the description of the model. To combat these problems and errors, the authors of the study have proposed a reproducibility scoring card, that should be implemented when publishing a model, either in the BioModels database or otherwise. It has to be noted, that this study has been done by the BioModel team themselves who have opened the discussion of the flaws in their database and science reproducibility in general. This shows, the need and desire of the community to make model sharing as easy as possible.