\begin{multicols}{2}

\section{What now?}

It is clear that there is a problem with computational models right now and while there are solutions proposed they still did not have enough traction to truly change this scientific field. This is where \greensloth{} comes into play. The main aspect of \greensloth{} is creating a bridge between scientists that create models and those that will actually try to use them for experimental work, as these two groups often do not coincide. To create this bridge, \greensloth{} will encompass many aspects of computational work that are already being tackled on as mentioned prior but with one key advantage. It will minimise its scope onto kinetic models of photosynthesis.

While models come in many different shapes and sizes, all showing different things, We believe it is important to first reduce the problem at hand to a manageable feat, by only encompassing a single, yet incredibly important, biochemical process. The importance of this process has also not escaped the eyes of the modelling community, as seen by the incredible amount of publication every year~\figref[]{fig:photosynthesisModelQuery}. Sifting through these amounts of publications is hard for both the modellers, that may want to expand their model with other interpretations, and the experimentalists that may want to build upon their work by using the new, popular and useful method of computational modelling.

The \greensloth{} project can be seperated into four main branches.

\begin{figure*}[!htb]
    \centering
    \includegraphics[height=0.7\textwidth, angle=90]{Figures/GreenSloth_Plan.pdf}
    \caption[]{}
    \label{fig:greenSlothPlan}
\end{figure*}

\subsection{The Promise}

Creating a model is no easy task.

\subsection{Model Creation Pipeline}

aihdfa

\subsection{Model Comparison \& Validation}

afgaf

\subsection{Dataset}

agagag

\subsection{Website}

Sharing models is no easy task. Only using publications can be tedious, as the code may not be available, or the model is not properly explained. Therefore, a centralised platform to collect and present models, would be beneficial to the scientific community. We believe that the easiest way to create such plattform is in a web-based format, where anyone can access it from anywhere. This platform will present models in a standardised format, that includes a summary, a table of model information, and the recreation of the publication figures. Furthermore, having a central location with all the models allows for easier comparison between them.

\newpage

    \subsection{A hole in science communication}

    Computational models have become a must-have in modern sciences, where the photosynthesis community is no exception. However, many different models have reached the "market" that show photosynthesis in their own way~\cite{stirbetPhotosynthesisBasicsHistory2020}. They are all different in their assumptions and their reasons for them. Some may be used to understand a specific part of the mechanism~\cite{matuszynskaMathematicalModelNonphotochemical2016, liImpactIonFluxes2021}. In contrast, others may be used to predict the behaviour and "output" of a plant~\cite{voncaemmererSteadystateModelsPhotosynthesis2013, zhuEphotosynthesisComprehensiveDynamic2013}. Some are built using kinetic knowledge, while others are based on stochastic processes. They all have key differences that make each unique and suitable for different purposes. Yet what most have in common is their lack of proper documentation and explanation.

    It is common and good in modern sciences to publish work in scientific journals. As experimental work has been the norm for as long as these publications have existed, the format is catered to that type of work. But with the explosive growth of computational work, the format aspect of these journals, like many other parts of our world and society, has not yet caught up. This has allowed much of the work in the modelling sector to be incomplete, irreproducible, and not adequately explained. This problem needs to be assessed and fixed, as many believe that working with models is the future of science.

    \subsection{Proposed solution}

    A proper solution would be a complete rework of the publication system to support digital work. Introducing a strict upload of code, data, and simulation protocols alongside the model description is only the beginning. You should also introduce the concept of a basic model formatting template so that most models can be read and understood by anyone due to their similar structures. This aspect has been tackled several times in many different ways, but has not gotten a real footing~\cite{sauroFAIRCUREGuidelines2025,huckaSystemsBiologyMarkup2003}. Furthermore, while the peer-review system is inherently flawed, to support computational work, reviewers need to be more adept in the field to understand errors in the code or model structure. These quick ideas only scratch the surface of what is required to fix the problem and may not be entirely feasible, so a more curated solution may be necessary initially.

    To make modelling a more novice-friendly field, a system has to be implemented that puts models right on the doorstep of anyone interested. While most of the time, this has been done through direct communication between computational experts and experimentalists, a more open system would be needed. Having the mind of an expert accessible to anyone and bridging the communication gap by "translating" it into a more novice-friendly format. This could be done by an open-access platform, where models are uploaded, explained and summarised, tested, and compared to other models. This would allow anyone interested in using a model to easily find one that fits their needs and understand how to use it properly. Attempts to this madness have already been tried on various occasions~\cite{malik-sheriffBioModels15Years2019}, but often with limited results~\cite{tiwariReproducibilitySystemsBiology2021}. These bleak prospects show the need for such a platform. Luckily, the building blocks to this solution have already been laid out by the master's thesis project GreenSloth (\url{https://greensloth.rwth-aachen.de/}).

    \subsection{Impact of the project}

    As the project is still under development, the full impact is hard to assess. However, word-of-mouth has already spread about the project. Interest has risen from both modellers who wish to contribute their models to the website, and experimentalists who want to use the platform to find models catered to their needs. As the project's popularity continues to rise, the platform needs to grow with it. This is where the proposed funding comes in to allow the development of the platform to continue and scale the work that comes with it.

    \subsection{Funding Uses}

    The funding would be used to pay for the development of the project, both in terms of time and fieldwork, to get more models on the platform and to understand what else is needed to help the community. A big focus will be put on science communication, particularly regarding figures and summaries, and it will be integrated into web design.

    The most significant part of the funding would be used to develop the science part of the project, the validation and testing of models. This allows a base quality control of models and enables a more direct comparison between models. While the master's thesis tied to the project has already tackled this aspect, it needs to be further expanded to alleviate the hurdle of understanding and choosing models for the novice. More understanding of what modellers do to build their models and what experimentalists need to use them properly is needed to make the platform a success, meaning working with both and gaining experience in both fields is key to the project. This includes creating User Test cases, creating a pipeline to help modellers upload their models in a reasonably similar format, and developing ways to facilitate working with the platform and the models themselves.

    A big understanding will be put into place to separate the project from the uploaded models. This platform will be used to help people understand the models, but not build them for them. This aspect will give the modellers their proper recognition, and will not try to rate them in any way. The project will solely be used to convey the knowledge of each model to the user, where they can make their own informed decision on which model to use. In no way will the project succumb to favoritism or bias towards any model; instead, it will strive to be as neutral as possible, following the scientific method and providing a detailed explanation of which parts were funded.

\end{multicols}