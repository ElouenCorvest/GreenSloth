\subsection{Preamble}

Photosynthesis is a fundamental process of the very existence of life on Earth~\cite{stirbetPhotosynthesisBasicsHistory2020, janssenPhotosynthesisForefrontSustainable2014,johnsonPhotosynthesis2016}. It provides the starting block for the food chain, and is directly responsible for the oxygen production that lead to the ozone layer. Additionally, it also proves to be a key player in human's more complex lives, as for example, around \qty{82}{\percent} of the world's primary energy consumption is still based on fossil fuels~\cite{FossilFuelShare}, which in turn are mostly made up of dead organic matter of photosynthetic organisms~\cite{pisupadtiFuelChemistry2003}. Therefore, photosynthesis has been a vital part of scientific research for centuries.

The basic principles of photosynthesis have been known since the 18\textsuperscript{th} century~\cite{ingenhouszExperimentsVegetablesDiscovering1779}, giving rise to advanced techniques to study the process in more detail. A key reason for this strive to understand photosynthesis is the potential to use it as a key way to increase crop yield~\cite{murchieAgricultureNewChallenges2009}. While there has been ground-breaking discoveries done in the past, that have introduced the Green Revolution, those methods have long reached a plateau in terms of improvement~\cite{longCanImprovementPhotosynthesis2006}. Therefore, there is a need to find new ways to change photosynthesis, however, this time in a new light. Instead of focusing on what comes out of experiments, a view more directed into the inner workings of photosynthesis is needed. To be able to change a process, it is necessary to understand it. This is where mathematical modelling comes in.

Mathematical modelling has been a staple of sciences for centuries~\cite{banerjeeMathematicalModelingModels2021}. From using simple geometry for calculating the distance to the sun, to simulate neutron transport in nuclear fission~\cite{demaziereModellingNuclearReactor2020}. This method has not gone unnoticed in the world of photosynthesis research, and has given rise to a large variety of models. Some focus on specific parts of the process~\cite{liImpactIonFluxes2021,matuszynskaMathematicalModelNonphotochemical2016}, some on specific ways the photosynthesis machinery reacts to the environment~\cite{fuenteMathematicalModelSimulate2024}. Some depict the process in a very simplified manner~\cite{farquharBiochemicalModelPhotosynthetic1980,voncaemmererSteadystateModelsPhotosynthesis2013,lochockiWidelyUsedVariants2025a}, while others try to capture the process in all its complexity~\cite{zhuEphotosynthesisComprehensiveDynamic2013}. There are many different ways mathematical modelling has been used to understand photosynthesis, which can be seen by still growing influx of new publications over the years'~\figref{fig:intro-photosynthesis-models}.

\begin{figure}
    \centering
    \includegraphics[width=\textwidth]{Figures/bibsearch.pdf}
    \captionprof{Results of a bibliography search for photosynthesis models.}{A basic bibliography search was done to find the publications with the words "photosynthesis" and "model" in the title, between 1970 and 2025. On the left, the cumulative number of publications over the years is shown, with the new publication of the respective year are shown in green. On the right, the number of new publications per Year is shown. The two colors were chosen to be easily distinguishable and do not have any specific meaning. The data was obtained from the Web of Science database on the 14\textsuperscript{th} of February 2026. The query used was "TI=(("photosynthesis" OR "photosynthetic") AND ("model*" OR "modelling" OR "modeling" OR "simulation" OR "representation"))", and can be found here: \url{https://www.webofscience.com/wos/woscc/summary/0b793bdb-ab8b-456e-a5fb-085fc470f5a2-019f07a73b/relevance/1}}
    \label{fig:intro-photosynthesis-models}
\end{figure}

With all these different models and ways to see photosynthesis, one cannot point to a single one as the "best" one. Each model has its own advantages and disadvantages, often tailored to answer specific questions. On top of that, many models work on top of each other, taking inspiration or even entire structures. It may be done to improve a model, or to make it more accessible to a different audience. Sometimes, it may also be used to answer a different scientific question, that may not have been the intention of the original model. A strong example of this, is the \gls{fvcb} model~\cite{farquharBiochemicalModelPhotosynthetic1980}, a simple model that describes the \gls{A} as a function of \gls{vc}, \gls{vo}, and \gls{rlight}. Even through its simplicity, it has amassed a large amount of citations, also in non-modelling branches. The main field using the \gls{fvcb} model are obviously the plant sciences, but it has found its way into other fields, such as ecology, forestry, and more~\figref{fig:intro-fvcb-cits}. It has become so popular in fact, that many different versions have come to fruition. Some that take the original model as a starting point and add on more complexity~\cite{voncaemmererSteadystateModelsPhotosynthesis2013,bellasioGeneralisedDynamicModel2019}, or others that use it solely as a readout for \gls{A}~\cite{zhuEphotosynthesisComprehensiveDynamic2013}. It has become so popular in fact, that even misinterpretations of the original model are ingrained in photosynthesis modelling~\cite{lochockiWidelyUsedVariants2025a}.

\begin{figure}
    \centering
    \includegraphics[width=0.3\textwidth]{Figures/fvcb_analyse.pdf}
    \captionprof{Number of citations of the original \glsentryshort{fvcb} publication separated in categories.}{The citations of the original \glsentryfull{fvcb} model publication~\cite{farquharBiochemicalModelPhotosynthetic1980} were obtained from the Web of Science database on the 16\textsuperscript{th} of February 2026. These citations were separated into categories based on the Web of Science categories of the citing publication. The five most populated categories were taken, while the others were grouped into "Others".}
    \label{fig:intro-fvcb-cits}
\end{figure}

While this rise in popularity of photosynthesis models brings, on the one hand, more and more tools to understand photosynthesis, it also brings a lot of confusion. Models differing in their concept, basing their work on different assumptions, and sometimes even misinterpreting starting blocks, can make it hard to see through the web of models. While it is not fair, to say that the waters of photosynthesis modelling are polluted, it is fair to say that there exists a problem of clarity. This is not only a problem for newcomers to the field, but also experienced researchers, who may find it hard to keep up with the ever-growing number of models. This problem of clarity, has already come into focus of the scientific community and it is not limited to photosynthesis modelling. Therefore, some proposed solutions already exist. Three key problems have been identified and solutions proposed: \hyperref[sec:intro-model-creation]{1) Model Creation}, \hyperref[sec:intro-model-presentation]{2) Model Presentation}, and \hyperref[sec:intro-model-sharing]{3) Model Sharing}. While all three are very intertwined, a clear distinction can be made between them.

\subsection{Current Problems and Solutions}

\subsubsection{Model Creation}\label{sec:intro-model-creation}

Model Creation, comes in many different ways, therefore the transparency of the process is vital. Some may be written from scratch using Python or Matlab, while others may be built using more specialized software for modelling, such as \verb|mxlpy|~\cite{vanaalstMxlPyPythonPackage2025}, \verb|COPASI|~\cite{hoopsCOPASICOmplexPAthway2006}, or \verb|Tellurium|~\cite{choiTelluriumExtensiblePythonbased2018}. Whatever the method used, one problem persists throughout all of them: the annotation and documentation of the model is based on the author. This means, that great care needs to be taken to make sure that the model is reproducible and understandable by others. This is not only a problem of the model description, but also of the code itself. As computational methods adapt and evolve, the code may become outdated and hard to read. There have been attempts to solve this problem, one of the most notable ones being the \gls{sbml}~\cite{huckaSystemsBiologyMarkup2003}. 

It breaks down the concept of a model into its most basic components, such as the compartment, species, reaction, and so on~\cite{huckaSystemsBiologyMarkup2003}. This creates a schematic overview of the model system which makes it much easier to be understood by different languages. Just as a schematic overview of photosynthesis is used in school books to explain that \gls{o2} is produced from water and not \gls{co2}, \gls{sbml} provides a schematic overview of the model so different software can understand it better. This simplification of complex systems

It does not provide ways to use the model, but it does provide a way to represent it in a way, that many different software can read it. Just like a schematic overview of photosynthesis is used in school books to explain that \gls{o2} is produced from water and not \gls{co2}, \gls{sbml} provides a schematic overview of the model so different software can understand it better.


\subsubsection{Model Presentation}\label{sec:intro-model-presentation}

\subsubsection{Model Sharing}\label{sec:intro-model-sharing}